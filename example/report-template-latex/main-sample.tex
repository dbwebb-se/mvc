\documentclass[oneside]{book}
%\documentclass{article}
\usepackage[utf8]{inputenc}
\usepackage[swedish]{babel}
\usepackage{biblatex}
\usepackage{csquotes}
\addbibresource{references.bib}

\setlength{\parskip}{1em}

\usepackage{hyperref}
\hypersetup{
    colorlinks=true,
    linkcolor=blue,
    filecolor=magenta,
    urlcolor=blue,
}


\begin{document}

\begin{titlepage}
    \begin{center}
        \vspace*{1cm}

        \Huge
        \textbf{Min Kursrapport}

        \vspace{0.5cm}
        \LARGE
        DV1610\\
        Objektorienterade webbteknologier\\
        (mvc)

        \vspace{0.5cm}
        \LARGE
        Förnamn Efternamn\\
        min-epost@some.where\\

        \vspace{0.5cm}
        \LARGE
        v1.0.0\\
        Mars 16, 2021

        \vfill

        \vspace{2cm}
        \Large
        Läsperiod 4\\
        Våren 2021

        \vspace{0.5cm}
        \Large
        Institutionen för datavetenskap (DIDA)\\
        Blekinge Tekniska Högskola (BTH)\\
        Sverige

    \end{center}
\end{titlepage}

\tableofcontents



\chapter{Objektorientering}

Här skriver du din redovisningstext för kmom01.

\section{Rubriker}

Använd gärna rubriker, vid behov, för att dela in din text och göra den mer tydlig och lättläst.

\section{Referenser}

Använd gärna referenser och bygg upp en referenslista som du använder genom kursen. En sådan lista kan vara bra att ha och titta tillbaka på när du letar efter resurser. Här följer ett exempel på hur du kan referera i text till en referens.

Skolverkets statistik för 2019/2020 \cite{skolverket_2020} rörande könsfördelning på gymnasiets högskoleförberedande Teknikprogram påtalar att endast 2 av 10 tjejer väljer den tekniska banan. Det är endast Teknikprogrammet som har en övervikt av killar. Naturprogrammet och Ekonomiprogrammet har en jämn könsfördelning och programmen Humanistiska och Samhällsvetenskap har en tydlig övervikt av tjejer. Möjligen är könsfördelningen i de högskoleförberedande gymnasieprogrammen ett tecken på att man kan förvänta sig en liknande könsfördelningen i de tekniska högskoleprogrammen.



\chapter{Controller}

Här skriver du din redovisningstext för detta kursmoment.



\chapter{Enhetstestning}

Här skriver du din redovisningstext för detta kursmoment.



\chapter{Ramverk}

Här skriver du din redovisningstext för detta kursmoment.



\chapter{Autentisering}

Här skriver du din redovisningstext för detta kursmoment.



\chapter{To be defined}

Här skriver du din redovisningstext för detta kursmoment.



\chapter{Projekt \& Examination}

Här skriver du din redovisningstext för detta avslutande kursmoment.

\section{Projektet}

Här skriver du din redovisningstext rörande projektet.

\section{Avslutningsvis}

Här skriver du de avslutande orden om kursen.



\newpage
\printbibliography

\end{document}
